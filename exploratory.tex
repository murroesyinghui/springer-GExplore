%!TEX root=./paper.tex
\subsection{Exploratory Graph Analysis (Davide)} 

\stitle{Example-based approaches.} 
One of the earliest attempts to bring examples as a query method is query-by-example~\citep{zloof1975query}.
The main idea was to help the user in the query formulation, allowing her to specify the shape of the results in terms of templates for tuples, i.e., examples.
Query-by-example has been lately revisited, and the use of examples have found application in graph data as well. 
The definition of example has transformed from a mere template to the representative of the intended results the user would like to have.
\emph{Examples} for a graph $G = (V,E)$ can be either \emph{set of nodes} $Q \subseteq V$, \emph{tuple} of nodes $(v_1, ..., v_n) \in V^n$ or \emph{subgraphs} $Q \subseteq G$, where with an abuse of notation $\subseteq$ indicates both set and structural inclusion. 


\dnote{Describe seed-based approaches}

\dnote{Describe Exemplar queries~\citep{mottin2014exemplar}, graph Query by example~\citep{jayaram2015querying}, and PANDA~\citep{xie2017panda}}


%, where $\ell : V \cup E \to \mathcal{L}$ is a labelling function on a set of labels $\mathcal{L}$ 



% \begin{enumerate}[label={\Roman*.}]
% 	\item \textbf{Introduction and motivation} (10 min)
% 		\begin{itemize}%[label={\bf\arabic*.}]
% 			\item Necessity of data exploration
% 			\item Lack of graph exploration due to the complexity of graph data
% 			\item Requirements for graph exploration
% 			\item Application for graph exploration systems. %to recommending systems, graph databases, graph visualization. 
% 		\end{itemize}
% 	\item \textbf{Data Exploration Taxonomy} (20 min) 
% 		\begin{enumerate}[label={\bf\arabic*.}]
% 			\item \textbf{Exploratory Graph Analysis}: approximate queries and queries-by-example
% 			\item \textbf{Refinement of Query Results}:\\ query reformulation and refinement, top-k results, skyline queries
% 			\item \textbf{Focused Graph Mining}: personalized queries, focused queries, data clustering
% 		%%% EXTENDED version with references if needed
% 		% 	\item \textbf{Exploratory queries}
% 		% 		\begin{itemize}
% 		% 			\item Approximate queries~\cite{qarabaqi2016merlin}
% 		% 			\item By example paradigm~\cite{zloof1975query,dimitriadou2014explore}
% 		% 		\end{itemize}
% 		% 	\item \textbf{Results presentation}
% 		% 		\begin{itemize}
% 		% 			\item Recommendation~\cite{drosou2013ymaldb,jiang2015snaptoquery}
% 		% 			\item Reformulation and refinement~\cite{mottin2013probabilistic,basu2008minimum,mishra2009interactive,chapman2009not}
% 		% 			\item Top-k results~\cite{arai2009anytime,chaudhuri1999evaluating,bruno2002top}
% 		% 		\end{itemize}
% 		% 	\item \textbf{Focused analytics}~\cite{stefanidis2010perk,arvanitis2012towards}
% 		\end{enumerate}
% 	\item \textbf{User-driven Graph Exploration} %(1h 40 min)
% 		\begin{enumerate}[label={\bf\arabic*.},start=0]
% 		\item \textbf{Background} (10 min)
% 			\begin{itemize}
% 				\item Graph models and terminology
% 				\item (Sub)graph isomorphism, graph edit distance
% 				\item Frequent subgraph mining
% 				\item Graph clustering and community detection
% 			\end{itemize}
\dnote{DAVIDE: Just a placeholder for the content.}




\stitle{Exploratory Graph Analysis} 
			\begin{itemize}
				\item Approximate search:
					\begin{itemize}
						\item Structural preserving: homomorphism~\cite{fan2010graph}, strong simulation~\cite{ma2014strong}
						\item Incompletely specified patterns~\cite{yang2014slq,khan2013nema,yuan2012efficient}
					\end{itemize}
				\item By example paradigm:
					\begin{itemize}
						\item Graph query by example~\cite{mottin2014exemplar,jayaram2015querying}	
						\item Learning paths~\cite{bonifati2014learning} 
					\end{itemize}
			\end{itemize}
\stitle{Refinement of Graph Query Results}
			\begin{itemize}
				\item Reformulation and refinement:
					\begin{itemize}
						\item Graph Query Reformulation with diversity~\cite{mottin2015graph}
						\item Why-empty and Why-so-many results~\cite{vasilyeva2016answering}
						\item Result summarization~\cite{ranu2014answering,wu2013summarizing}
					\end{itemize}
				\item Top-k results: 
					\begin{itemize}
					 	\item Diversified Top-k Graph Pattern Matching~\cite{fan2013diversified}
					 	\item Learning to rank from user-feedback~\cite{su2015exploiting}
					 	\item Top-K interesting subgraph discovery in information networks~\cite{gupta2014top,jin2015querying}
					 \end{itemize}
				\item Skyline queries~\cite{zheng2014efficient,zou2010dynamic}  
			\end{itemize}
% 		\item \textbf{Focused Graph Mining} (35 min) 
% 			\begin{itemize}
% 				\item Focused Graph clustering and outlier detection:
% 					\begin{itemize}
% 						\item Focused clustering providing seed nodes~\cite{perozzi2014focused,iglesias2015efficient,kloumann2014community}
% 						\item Query-driven outlier detection~\cite{gupta2014local,zhuang2014mining}
% 					\end{itemize}
% 				\item Space restriction methods: 
% 					\begin{itemize}
% 						\item Ego-networks~\cite{epasto2015ego} and local community detection~\cite{ruan2013efficient,staudt2014detecting}
% 						\item Center-piece subgraphs~\cite{tong2006center}
% 						\item Query-driven graph summarization~\cite{zhang2010discovery}
% 					\end{itemize}
% 				\item Reweighting graphs: WIGM~\cite{yang2012wigm}	
% 			\end{itemize}
% 	\end{enumerate}
% 	\item \textbf{Real-world use case} (15 min) 
% 		\begin{itemize}
% 			\item Linked Data graphs: exploration, query refinement, and mining
% 		\end{itemize}
% 	\item \textbf{Open challenges} (20 min) 
% 		\begin{itemize}%[label={\bf\arabic*.}]
% 			% \item \textbf{Applications} {\small [to be excluded in a 1.5 hours tutorial]}
% 			% 	\begin{itemize}
% 			% 		\item Better recommendations for targeted advertising 
% 			% 		\item Improved graph exploration tools
% 			% 		\item Supporting mobile devices (limited screen space)
% 			% 		\item Adaptive techniques for improved performance
% 			% 	\end{itemize}
% 			%\item \textbf{Open questions}
% 			%	\begin{itemize}
% 			\item Can we \emph{interactively} assist the user toward the retrieval of the correct answer?  
% 			\item Can we provide \emph{explanations} for the query results? 
% 			\item Is there a way to efficiently \emph{adapt} the analyses on-demand? 
% 			\item Can we integrate these techniques into current \emph{graph databases}?
% 			%	\end{itemize}
% 		\end{itemize}



