%!TEX root=./paper.tex
\subsection{Exploratory Graph Analysis (Davide)} 

\stitle{Example-based approaches} 

\noindent One of the earliest attempts to bring examples as a query method is query-by-example~\citep{zloof1975query}.
The main idea was to help the user in the query formulation, allowing her to specify the shape of the results in terms of templates for tuples, i.e., examples.
Query-by-example has been lately revisited, and the use of examples have found application in graph data. 
The definition of example has transformed from a mere template to the representative of the intended results the user would like to retrieve.
\emph{Examples} for a graph $G = (V,E)$ can be either \emph{set of nodes} $Q \subseteq V$, \emph{tuple} of nodes $(v_1, ..., v_n) \in V^n$, or \emph{subgraphs} $Q \subseteq G$, where with an abuse of notation $\subseteq$ indicates both set and structural inclusion. 

\eetitle{Set of nodes as examples.} Several analyses can be performed when a set of examples nodes $Q\subseteq V$ is provided; this includes the discovery of dense graph regions~\citep{gionis2015bump,ruchansky2015minimum}, central nodes~\citep{tong2006center}, and communities~\citep{staudt2014detecting,perozzi2014focused}.%, and targeted graph summarization~\citep{zhang2010discovery}. 
Discrepancy maximization in graphs~\citep{gionis2015bump} aims at finding a connected subgraph $G' = (V', E'), G' \subseteq G$ that maximizes the discrepancy $\delta(G'){=}\alpha|Q|{-}|V{\setminus}Q|, \alpha > 1$; in other words, $G'$ is a subgraph containing more nodes from $Q$ than other nodes. 
Discrepancy maximization is \NP-hard. 
On a similar line, the Center-piece subgraph problem (CEPS)~\citep{tong2006center} finds the maximum weight subgraph that contains at least $k$ query nodes and at most $b$ other nodes. 
A generalization of the CEPS problem, the minimum Wiener connector problem (MWC)~\citep{ruchansky2015minimum} computes dense regions of a graph potentially representing community structures, given a set of nodes as input. MWC aims at finding a subgraph of $G$ induced by $Q$, denoted as $G[Q]$, that minimizes the sum of the pairwise distances among the nodes in $Q$, i.e, $\argmin_{G[Q]}\sum_{\{u,v\}{}\in{}Q}{}d_{G[Q]}(u,v)$. 

Providing examples has also been used to detect communities~\citep{staudt2014detecting,perozzi2014focused}, where a community is a set of nodes, in which every node is connected to many more nodes inside the community than outside. 
One approach is to consider each node in $v \in Q$ as a representative of a different community~\citep{staudt2014detecting}. 
In graphs with attributes on nodes, FocusCO~\citep{perozzi2014focused} introduced a clustering algorithm that learns a similarity function among nodes, based on the attributes in the 
Formally, FocusCO takes an attributed graph $G = (V,E,F)$, where $F$ is a feature matrix $F \in \mathbb{R}^n\times f$ and $F_{i\cdot}$ represents the attribute vector for node $v_i$, and returns cluster that better represent the seed nodes $Q$.
In addition, FocusCO returns outliers that do not belong to any community identified so far.  

% Last, graph summarization can exploit the information on the example nodes to preserve only the relevant information in the nodes~\cite{zhang2010discovery}. 
% \dnote{Learning queries by example~\citep{bonifati2014learning}}


\eetitle{Node tuples as examples.}
In Graph Query by Example (GQBE)~\citep{jayaram2015querying} a tuple of nodes $(v_1, ..., v_n), v_i \in V^n$ is a representative of the result tuples. 
For instance, if the pair (\emph{Yahoo, Jerry Yang}) is provided, the result is expectedly other \emph{IT company, founder} pairs. 
The solution first computers \emph{minimal answer graph} connecting the query nodes and then find other subgraph isomorphic to the answer graph. 

\eetitle{Subgraphs as examples.}
Subgraphs are more expressive than nodes and tuples, hence they can be exploited to obtain more accurate results. 
In the case of graphs, the example can constitute an Exemplar Query~\citep{mottin2016exemplar}, thus $Q\subseteq G$ and a result is a subset of $G$ congruent to $Q$.
Exemplar Queries is a flexible paradigm that allows the definition of multiple congruence relations among the input example and the intended results. 
Moreover, it supports efficient retrieval of top-$k$ results. 

A generalization of exemplar queries, PANDA~\citep{xie2017panda}, studies partial topology-based network search. 
That is, to find the connections (paths) between structures node-label isomorphic to different user inputs. 
PANDA first materialize all isomorphic graphs, then groups them into connected components, and finally finds undirected shortest paths among them. 


%\dnote{DAVIDE: Maybe to expand a bit more.}


\stitle{Reformulation of Graph Queries}

\noindent Graph exploration entails that a user provided query on a graph is indefinite in nature, and therefore is likely to return either too few or too-many results. 
For this reason query reformulation techniques aim at modifying the query to lead the user to the intended result~\citep{mottin2015graph,hurtado2008query,islam2015efficient}. 
A \emph{reformulation} for a query $Q \subseteq G$ on a labeled graph $G = (V, E, \ell)$ with labeling function on nodes and edges $\ell$, is another query $Q'$ that is either a supergraph or a subgraph of $Q$, $Q' \subseteq Q$ or $Q' \supseteq Q$. Therefore, $Q'$ can be more specific (supergraph) or more generic (subgraph) than $Q$. 
Queries considered in graph query reformulation are usually subgraph isomorphism queries~\cite{lee2012depth}.  
Query reformulation has been proposed for collections of graphs, or \emph{graph databases}, as well as large graphs. 
%For simplicity the graph is considered unlabeled, even though the presented methods work mostly with labeled graphs. 
%the set of $\mathcal{R}(Q)=\{G' \subseteq G | G' \equiv Q \}$ with a

\eetitle{Query reformulation in graph databases.}
Query reformulation in graphs can apply to collection of graphs or \emph{graph databases} $\mathcal{D} = \{G_1, ..., G_n\}$, where each $G_i$ is a graph $G_i = (V_i, E_i, \ell_i)$. 
In such graph databases, the query $Q$ returns a subset $\mathcal{R}(Q) \subseteq \mathcal{D}$ containing the structure $Q$. 

Graph Query Reformulation (GQRef)~\citep{mottin2015graph} discovers query reformulations having small overlap and high coverage; thus ideally no result is missing from the reformulated queries, yet the reformulations have little redundancy. 
The objective is to find a set $R$ of $k$ reformulations with high coverage and diversity (i.e., small overlap), therefore maximizing $f(R) = |\bigcup_{Q \in R}\mathcal{R}(Q)|+ \lambda\sum_{Q_1,Q_2 \in R}|\mathcal{R}(Q_1) \cup \mathcal{R}(Q_2)|$. 
Maximizing $f$ is \NP-hard, but an approximate greedy solution can guarantee a $(1/2)$-approximation. 

Alternatively, AutoG~\citep{yi2017autog} proposes a top-$k$ reformulation approach for visual autocompletion of queries. 
AutoG returns the $k$ best relaxations according to a ranking function that favors a large number of results for each reformulation and large diversity. 
Such ranking function (excluding normalization) is $u(R) = \alpha \sum_{Q \in R}|\mathcal{R}(Q)|/|\mathcal{D}| + (1-\alpha) \sum_{Q_1, Q_2 \in R} \mathsf{dist}(Q_1, Q_2)$, where  $\alpha \in [0,1]$ strikes a balance between coverage and distance $\mathsf{dist}$ among queries.



\eetitle{Query reformulation in large graphs.}
Few works explicitly explore query reformulation large graphs.
Early attempts in this direction propose to enrich the SPARQL semantics with a \textsc{RELAX} clause that returns a set of reformulations following some predefined rules (such as expanding edges with a specific label)~\citep{hurtado2008query}.
In~\citep{arenas2014faceted} the SPARQL reformulations are extended to OWL logic, where the semantics of each reformulation is captured by the RDF taxonomy. 
The user can interactively accept or reject the possible expansions of a query such that the navigation is consistent with the rules. 

Query reformulations are also used to debug queries and understand why the query returned too-many or too-few answers~\citep{vasilyeva2016answering}. 
In order to explain the dependency between the number of results and the query~\citep{vasilyeva2016answering} first computes the \emph{maximum common subgraph} (MCS) among the graph $G$ and the query $Q$, and then use a \emph{differential graph} to profile the query by adding edges from the differential graph to the MCS. 
The maximum common subgraph is the largest connected graph $G'$ that is subgraph of both $Q$ and $G$. 

VIIQ~\cite{jayaram2015viiq} employs query logs to rank the possible reformulations of the query $Q$. 
VIIQ's ranking function generates \emph{correlation paths} among the sequence of reformulation in the current query $Q$ and the sequences in the query log. 
Among the candidate \emph{single} edges that can be added to $Q$ the ranking function proposes those that have a larger support in the query log. 

%\dnote{DAVIDE: Expand?}

% * VIIQ: Large graph, Query log, active and passive mode of candidate edge generation, scoring function based on correlation paths in the sequence in the query log, and ranking using prefixes and postfixes in those paths. 

\eetitle{Why-not queries in graphs.}
Why-not queries additionally require the user to provide a set of missing answers from the results of $Q$, and the system proposes a new query $Q'$ returning those results and the fewer number of irrelevant results. 
The only work in this direction is~\cite{islam2015efficient}. 
In order to return the why-not queries, the maximum common subgraph (MCS) among the query $Q$ and the missing answers $\mathcal{D}^-$ is computed. The best query is found minimizing the maximum distance among $Q'$ and the missing answers $\mathcal{D}^-$ as $\argmin_{Q'\in \mathcal{D}} \max\{\Delta(Q',G)|G \in D^{-}\})$, where given $G_1 = (E_1,V_1, \ell_1), G_2 = (E_2,V_2, \ell_2)$, $\Delta(G_1, G_2) = |E_1| + |E_2| - 2 |\mathsf{MCS}(G_1,G_2)|$. 


%% SKYLINE? Do we add this? 
% \eetitle{Skyline queries on graphs.}
% Skyline queries~\cite{zheng2014efficient,zou2010dynamic}  












% \stitle{Exploratory Graph Analysis} 
% 			\begin{itemize}
% 				\item Approximate search:
% 					% \begin{itemize}
% 					% 	\item Structural preserving: homomorphism~\cite{fan2010graph}, strong simulation~\cite{ma2014strong}
% 					% 	\item Incompletely specified patterns~\cite{yang2014slq,khan2013nema,yuan2012efficient}
% 					% \end{itemize}
% 				\item By example paradigm:
% 					\begin{itemize}
% 						\item Graph query by example~\cite{mottin2014exemplar,jayaram2015querying}	
% 						\item Learning paths~\cite{bonifati2014learning} 
% 					\end{itemize}
% 			\end{itemize}
% \stitle{Refinement of Graph Query Results}
% 			\begin{itemize}
% 				\item Reformulation and refinement:
% 					\begin{itemize}
% 						\item Graph Query Reformulation with diversity~\cite{mottin2015graph}
% 						\item Why-empty and Why-so-many results~\cite{vasilyeva2016answering}
% 						\item Result summarization~\cite{ranu2014answering,wu2013summarizing}
% 					\end{itemize}
% 				\item Top-k results: 
% 					\begin{itemize}
% 					 	\item Diversified Top-k Graph Pattern Matching~\cite{fan2013diversified}
% 					 	\item Learning to rank from user-feedback~\cite{su2015exploiting}
% 					 	\item Top-K interesting subgraph discovery in information networks~\cite{gupta2014top,jin2015querying}
% 					 \end{itemize}
% 				\item Skyline queries~\cite{zheng2014efficient,zou2010dynamic}  
% 			\end{itemize}



