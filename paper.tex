%%%%%%%%%%%%%%%%%%%% author.tex %%%%%%%%%%%%%%%%%%%%%%%%%%%%%%%%%%%
%
% template for Encyclopedia articles
%
%%%%%%%%%%%%%%%% Springer %%%%%%%%%%%%%%%%%%%%%%%%%%%%%%%%%%


% RECOMMENDED %%%%%%%%%%%%%%%%%%%%%%%%%%%%%%%%%%%%%%%%%%%%%%%%%%%
% RECOMMENDED %%%%%%%%%%%%%%%%%%%%%%%%%%%%%%%%%%%%%%%%%%%%%%%%%%%
\documentclass[graybox, natbib, nosecnum, twocolumn]{svmult}
\bibpunct{(}{)}{;}{a}{}{,} % suppress commas between author-names and year

% choose options for [] as required from the list
% in the Reference Guide

\usepackage{mathptmx}       % selects Times Roman as basic font
\usepackage{helvet}         % selects Helvetica as sans-serif font
\usepackage{courier}        % selects Courier as typewriter font
\usepackage{type1cm}        % activate if the above 3 fonts are
                            % not available on your system

\usepackage{makeidx}         % allows index generation
\usepackage{graphicx}        % standard LaTeX graphics tool
                             % when including figure files
\usepackage{multicol}        % used for the two-column index
\usepackage[bottom]{footmisc}% places footnotes at page bottom
\usepackage[normalem]{ulem}	% for strike-through of text with \sout{}  
\usepackage{hyperref}  %for hyperlinks
\usepackage{soul}   % for high-lighting of text
\usepackage{xspace}
\usepackage{enumitem} 		% For customizing lists
\usepackage{amsfonts}		% using of mathbb fonts
\usepackage{amsmath}
% \usepackage{amsthm}


%%%%%%%%%%%%%%%%%%%%%%%%%%%%%%%%%%%%%%%%%%%%%%%%%%%%%%%%%%%%%%%%%%%%%%%%%%%%%%%%%%%%%%%%%

\newcommand{\eat}[1]{}
\newcommand{\stitle}[1]{\vspace{1.6ex}\noindent{\bf #1}}
\newcommand{\eetitle}[1]{\vspace{0.8ex}\noindent{\underline{\em #1}}}
\newcommand{\etitle}[1]{\vspace{0.6ex}\noindent{{\em #1}}}
\newcommand{\ie}{\emph{i.e.,}\xspace}
\newcommand{\NP}{\kw{NP}}
\newcommand{\dist}{\kw{dist}}
\newcommand{\len}{\kw{len}}
\newcommand{\kw}[1]{{\ensuremath {\mathsf{#1}}}}
\newcommand{\MST}{\kw{MST}}
\newcommand{\kws}{\kw{KWS}}
\newcommand{\eg}{\emph{e.g.,}\xspace}
\newcommand{\dnote}[1]{{\color{red}#1}}


\DeclareMathOperator*{\argmax}{\arg\max}
\DeclareMathOperator*{\argmin}{\arg\min}



\begin{document}

\title*{Graph Exploration and Search}
% Use \titlerunning{Short Title} for an abbreviated version of
% your contribution title if the original one is too long
\author{Yinghui Wu\\Davide Mottin}
% Use \authorrunning{Short Title} for an abbreviated version of
% your contribution title if the original one is too long
\institute{Yinghui Wu, Washington State University, \email{yinghui@eecs.wsu.edu}}
%
\institute{Hasso Plattner Institute, \email{davide.mottin@hpi.de}}
% Use the package "url.sty" to avoid
% problems with special characters
% used in your e-mail or web address
%
\maketitle

%\section{Synonyms}
%Provide synonyms to the article title.

\section{Definition}

Exploratory methods have been originally introduced as a mean to extract knowledge from relational data without knowing what to search~\cite{idreos2015overview}. 
Recently, \emph{graph exploration} has been introduced to perform exploratory analyses on graph-shaped data~\cite{mottin2017graph}. 
Graph exploration aims at mitigating the access to the data to the user, even if such user is a novice. 

Algorithms for graph exploration assume the user is not able to completely specify the object of interest with  a structured query, like a SPARQL~\cite{prud2006sparql} query, but rather expresses the need with a simpler, more ambiguous language.

This asymmetry between the rigidity of structured queries and ambiguity of the user has inspired the study of approximate, flexible, and example-based methods. 


%Main Text
%%% Explain the main techniques
\section{Overview}

The research on graph exploration has revolved around three main pillars: \emph{keyword graph queries}, \emph{exploratory graph analysis}, and \emph{refinement of query results} 

% We abstracted user-driven graph exploration properties from techniques proposed in the literature and defined such a unified taxonomy. 


\stitle{Keyword Graph Queries} \dnote{(Yinghui)} 

\stitle{Exploratory Graph Analysis} entails the process of casting an incomplete or imperfect pattern query to let the system find the closest match. Such exploratory analysis may return a huge number of results, e.g., structures matching the pattern. Thus, the system is required to provide intelligent support. One such strategy is the well known query-by-example paradigm, in which the user provides the template for the tuples and let the system infer the others. 

\stitle{Refinement of Graph Query Results} is needed to deal with the overwhelming amount of results that is typical in subgraph processing. It includes approaches designed to present comprehensive result sets to the user or intermediate results that can be refined further. 
%Instantiations of this kind are graph summaries, top-k methods, query reformulation, and skyline queries. 
	% \item [Focused Graph Mining] guides the users to a specific portion of the graph they are interested in. It requires the user to provide feedback in the process to restrict the computation to some portion of the graph. Ego-networks mining belongs to this strategy, since the user search is limited to a particular area of the graph and the algorithms focus on that specific area. 


\section{Key Research Findings}

\subsection{Graph Search with Keywords (Yinghui)} 

Keyword search (\kws) has been routinely used to 
explore and understand graph data~\citep{wang2010survey,bao2015exploratory,yahya2016exploratory,mottin2017graph}. 
(1) A keyword query $Q$ is a set of terms $\{t_1, ..., t_n\}$. 
Given graph $G$=$(V, E)$ and a term $t_i$, a match function 
determines a set of {\em content nodes} $V(t_i)\subseteq V$ that match $t_i$. 
(2) An answer 
$G_Q$ of a keyword query $Q$ is a subgraph of $G$ that contains at least 
one node from $V(t_i)$ for each $t_i\in Q$. 
In keyword-based graph exploration, 
a function $F(G_Q)$ is used to quantify the 
cost of connecting all content nodes in $G$. 
Answers with smaller cost have higher quality. 

Given a node pair $(u, v)$ 
in graph $G$, (1) the distance from $u$ to $v$,
denoted as $\dist(u,v)$, is the sum of edge weight 
on the shortest path from $u$ to $v$; and  
(2) $\len(u,v)$ denotes the length of 
the shortest path from $u$ to $v$. 
There are three common classes of \kws queries. 

\vspace{.5ex}
\etitle{Distinct root-based KWS}~\citep{kacholia2005bidirectional,he2007blinks}. 
These query defines $G_Q$ as a minimal rooted tree 
which (1) contains a distinct root node $v_r$, and 
at least a content node $v_i\in V(t_i)$ as a leaf for each $t_i\in Q$; 
and (2) $\len(v_r, v_i)\leq r$ for a predefined hop bound $r$, 
for each leaf $v_i$.  Here, $G_Q$ is minimal if 
no subtree of $G_Q$ is an answer of $Q$ in $G$. 
The function $F(G_Q)$ is defined as 
$F(G_Q) = \sum_{t_i\in Q}\dist(v_r, v_i)$, 
where $v_i$ ranges over the content nodes. 

The answers of such queries can be found in  
$O(|Q|(|V|\log|V|+|E|))$ time (cf.~\citep{yu2010keyword}). 

\etitle{Steiner tree-based KWS}~\citep{bhalotia2002keyword}. 
A Steiner tree-based query differs 
from its distinct root-based counterparts 
in that it uses a different cost function $F(G_Q)$, which is defined as 
$\sum_{e\in G_Q} w(e)$, 
\ie the total weight of the edges in the Steiner tree $G_Q$. 

%%%%%%%%%
It is \NP-hard~\citep{yu2010keyword} to evaluate a Steiner tree-based query  
by computing a minimum weighted Steiner tree (\MST, a known \NP-hard problem~\citep{ding2007finding}). 
Both exact~\citep{ding2007finding} and 
approximation algorithms~\citep{byrka2013steiner} 
have been developed to evaluate such queries. 

\vspace{.5ex}
\etitle{Steiner graph-based KWS}~\citep{li2008ease,kargar2011keyword}. 
Finding $G_Q$ as graphs rather than trees may be more helpful. For Steiner graph-based queries with a number $r$,  
$G_Q$ is a Steiner graph that contains content and Steiner nodes (\ie nodes 
on shortest paths between two content nodes), with either radius bounded by $r$ 
(\ie r-radius Steiner graph~\citep{li2008ease}), or distance between 
any two content nodes bounded by $r$ (\ie $r$-clique~\citep{kargar2011keyword}). 
For an answer $G_Q$ with nodes $\{v_1, \ldots, v_n\}$,  its cost $F(G_Q)$ 
is computed as  
$\sum_{i\in[1, n]}\sum_{j\in[i+1,n]}\dist(v_i, v_j)$, 
\ie the total pairwise distances of the 
content nodes in $G_Q$. 
Here, the distances are typically 
induced by undirected shortest paths.  

It is in general \NP-hard 
to evaluate Steiner graph-based queries~\citep{yu2010keyword, kargar2011keyword}. 
Approximate algorithms are developed for such queries to 
find $r$-radius graphs~\citep{li2008ease}
and $r$-cliques~\citep{kargar2011keyword}. 

\stitle{Keyword query suggestion}. 
Keyword query suggestion and its variants 
(\eg query refinement~\cite{mishra2009interactive} and reformulation~\citep{yao2012keyword}) 
have been studied to suggest new queries that better describe search intent 
for graph exploration. 
Most prior work adopts information retrieval techniques that make use of 
query logs and user feedback~\citep{carpineto2012survey,cao2008context}.
Keyword query suggestion for graphs has been studied in ~\citep{tran2009top} by suggesting structured queries computed from the keywords over RDF data. In order to summarize the results of \kws over structured data, tag clouds~\citep{koutrika2009data} discovers the most significant words retrieved as a part of the initial results. 
Provable quality guarantees of the answers are not addressed in these work. 

Query suggestion has been studied 
for XML data~\citep{zeng2014breaking}, 
with a focus on coping with missed matches. An approximate 
answer of original query $Q$ is computed by expanding 
its answer with content nodes of the same type, and 
new keywords are suggested to replace those 
without match in $Q$. 

%\subsection{Graph Search with Schemaless Queries (Yinghui)} 

 

%!TEX root=./paper.tex
\subsection{Exploratory Graph Analysis (Davide)} 

\stitle{Example-based approaches} 

\noindent One of the earliest attempts employing examples as search conditions is query-by-example~\citep{zloof1975query}.
The main idea was to help the user in the query formulation, allowing her to specify the shape of the results in terms of templates for tuples, i.e., examples.
Query-by-example has been lately revisited, and the use of examples have found application in graph data. 
The definition of example has transformed from a mere template to the representative of the intended results the user would like to retrieve.
\emph{Examples} for a graph $G = (V,E)$ can be either \emph{set of nodes} $Q \subseteq V$, \emph{tuple} of nodes $(v_1, ..., v_n) \in V^n$, or \emph{subgraphs} $Q \subseteq G$, where with an abuse of notation $\subseteq$ indicates both set and structural inclusion. 

\eetitle{Set of nodes as examples.} Several analyses can be performed when a set of examples nodes $Q\subseteq V$ is provided; this includes the discovery of dense graph regions~\citep{gionis2015bump,ruchansky2015minimum}, central nodes~\citep{tong2006center}, and communities~\citep{staudt2014detecting,perozzi2014focused}.%, and targeted graph summarization~\citep{zhang2010discovery}. 
Discrepancy maximization in graphs~\citep{gionis2015bump} aims at finding a connected subgraph $G' = (V', E'), G' \subseteq G$ that maximizes the discrepancy $\delta(G'){=}\alpha|Q|{-}|V{\setminus}Q|, \alpha > 1$; in other words, $G'$ is a subgraph containing more nodes from $Q$ than other nodes. 
Discrepancy maximization is \NP-hard. 
On a similar line, the Center-piece subgraph problem (CEPS)~\citep{tong2006center} finds the maximum weight subgraph that contains at least $k$ query nodes and at most $b$ other nodes. 
A generalization of the CEPS problem, the minimum Wiener connector problem (MWC)~\citep{ruchansky2015minimum} computes dense regions of a graph potentially representing community structures, given a set of nodes as input. MWC aims at finding a subgraph of $G$ induced by $Q$, denoted as $G[Q]$, that minimizes the sum of the pairwise distances among the nodes in $Q$, i.e, $\argmin_{G[Q]}\sum_{\{u,v\}{\in}Q}{}d_{G[Q]}(u,v)$. 

Examples have also been used as a mean to detect communities~\citep{staudt2014detecting,perozzi2014focused}, where a community is a set of nodes, in which every node is connected to many more nodes inside the community than outside. 
One approach is to consider each node in $v \in Q$ as a representative of a different community~\citep{staudt2014detecting}. 
In graphs with attributes on nodes, FocusCO~\citep{perozzi2014focused} introduced a clustering algorithm that learns a similarity function among nodes, based on the attributes in the examples.
Formally, FocusCO takes an attributed graph $G = (V,E,F)$, where $F$ is a feature matrix $F \in \mathbb{R}^n\times f$ and $F_{i\cdot}$ represents the attribute vector for node $v_i$, and returns cluster that better represent the seed nodes $Q$.
In addition, FocusCO returns outliers that do not belong to any community identified so far.  

% Last, graph summarization can exploit the information on the example nodes to preserve only the relevant information in the nodes~\cite{zhang2010discovery}. 
% \dnote{Learning queries by example~\citep{bonifati2014learning}}


\eetitle{Node tuples as examples.}
In Graph Query by Example (GQBE)~\citep{jayaram2015querying} a tuple of nodes $(v_1, ..., v_n), v_i \in V^n$ is a representative of the result tuples. 
For instance, if the pair (\emph{Yahoo, Jerry Yang}) is provided, the result is expectedly other \emph{IT company, founder} pairs. 
The solution first computes the \emph{minimal answer graph} connecting the query nodes and then finds other subgraphs isomorphic to the answer graph. 

\eetitle{Subgraphs as examples.}
Subgraphs are more expressive than nodes and tuples, hence they can be exploited to obtain more accurate results. 
In the case of graphs, the example can constitute an Exemplar Query~\citep{mottin2016exemplar} $Q\subseteq G$ and a result is a subset of $G$ congruent to $Q$.
Exemplar Queries is a flexible paradigm that allows the definition of multiple congruence relations among the input example and the intended results, such as isomorphism or strong simulation~\citep{ma2014strong}. 
Moreover, it supports efficient retrieval of top-$k$ results. 

A generalization of exemplar queries, PANDA~\citep{xie2017panda}, studies partial topology-based network search. 
That is, to find the connections (paths) between structures node-label isomorphic to different user inputs. 
PANDA first materialize all isomorphic graphs, then groups them into connected components, and finally finds undirected shortest paths among them. 


%\dnote{DAVIDE: Maybe to expand a bit more.}


\stitle{Reformulation of Graph Queries}

\noindent Users with exploratory intent typically provide indefinite queries that are likely to return either too few or too-many results. 
For this reason query reformulation techniques aim at modifying the query to lead the user to the intended result~\citep{mottin2015graph,hurtado2008query,islam2015efficient}. 
A \emph{reformulation} for a query $Q \subseteq G$ on a labeled graph $G = (V, E, \ell)$ with labeling function on nodes and edges $\ell$, is another query $Q'$ that is either a supergraph $Q' \subseteq Q$ or a subgraph $Q' \supseteq Q$ of $Q$. Therefore, $Q'$ can be more specific (supergraph) or more generic (subgraph) than $Q$. 
Queries considered in graph query reformulation are usually subgraph isomorphism queries~\cite{lee2012depth}.  
Query reformulation has been proposed for collections of graphs, or \emph{graph databases}, as well as \emph{large graphs}. 
%For simplicity the graph is considered unlabeled, even though the presented methods work mostly with labeled graphs. 
%the set of $\mathcal{R}(Q)=\{G' \subseteq G | G' \equiv Q \}$ with a

\eetitle{Query reformulation in graph databases.}
Query reformulation in graphs can apply to collection of graphs or \emph{graph databases} $\mathcal{D} = \{G_1, ..., G_n\}$, where each $G_i$ is a graph $G_i = (V_i, E_i, \ell_i)$. 
In such graph databases, the query $Q$ returns a subset of the collection $\mathcal{R}(Q) \subseteq \mathcal{D}$ containing the structure $Q$. 

Graph Query Reformulation (GQRef)~\citep{mottin2015graph} discovers query reformulations having small overlap and high coverage; thus ideally no result is missing from the reformulated queries, yet the reformulations have little redundancy. 
The objective is to find a set $R$ of $k$ reformulations with high coverage and diversity (i.e., small overlap)  maximizing $f(R) = |\bigcup_{Q \in R}\mathcal{R}(Q)|+ \lambda\sum_{Q_1,Q_2 \in R}|\mathcal{R}(Q_1) \cup \mathcal{R}(Q_2)|$. 
Maximizing $f$ is \NP-hard, but an approximate greedy solution can guarantee a $(1/2)$-approximation. 

Alternatively, AutoG~\citep{yi2017autog} proposes a top-$k$ reformulation approach for visual autocompletion of queries. 
AutoG returns the $k$ best relaxations according to a ranking function that favors a large number of results for each reformulation and large diversity. 
Such ranking function (excluding normalization) is $u(R) = \alpha \sum_{Q \in R}|\mathcal{R}(Q)|/|\mathcal{D}| + (1-\alpha) \sum_{Q_1, Q_2 \in R} \mathsf{dist}(Q_1, Q_2)$, where  $\alpha \in [0,1]$ strikes a balance between coverage and a user-defined distance $\mathsf{dist}$ among queries.



\eetitle{Query reformulation in large graphs.}
Few works explicitly explore query reformulation in large graphs.
Early attempts in this direction propose to enrich the SPARQL semantics with a \textsc{RELAX} clause that returns a set of reformulations following some predefined rules (such as expanding edges with a specific label)~\citep{hurtado2008query}.
In~\citep{arenas2014faceted} the SPARQL reformulations are extended to OWL logic, where the semantics of each reformulation is captured by the RDF taxonomy. 
The user can interactively accept or reject the possible expansions of a query such that the navigation is consistent with the rules. 

Query reformulations are also used to debug queries and understand why the query returned too-many or too-few answers~\citep{vasilyeva2016answering}. 
In order to explain the dependency between the number of results and the query~\citep{vasilyeva2016answering} first computes the \emph{maximum common subgraph} (MCS) among the graph $G$ and the query $Q$, and then use a \emph{differential graph} to profile the query by adding edges from the differential graph to the MCS. 
The maximum common subgraph is the largest connected graph $G'$ that is subgraph of both $Q$ and $G$. 

VIIQ~\cite{jayaram2015viiq} employs query logs to rank the possible reformulations of the query $Q$. 
VIIQ's ranking function generates \emph{correlation paths} among the sequence of reformulations in the current query $Q$ and the sequences in the query log. 
Among the candidate \emph{single} edges that can be added to $Q$ the ranking function proposes those that have a larger support in the query log. 

%\dnote{DAVIDE: Expand?}

% * VIIQ: Large graph, Query log, active and passive mode of candidate edge generation, scoring function based on correlation paths in the sequence in the query log, and ranking using prefixes and postfixes in those paths. 

\eetitle{Why-not queries in graphs.}
Why-not queries additionally require the user to provide a set of missing answers from the results of $Q$, and the system proposes a new query $Q'$ returning those results and the fewer number of irrelevant results. 
The only work in this direction is~\cite{islam2015efficient}. 
In order to return the why-not queries, the maximum common subgraph (MCS) among the query $Q$ and the missing answers $\mathcal{D}^-$ is computed. The best query is found minimizing the maximum distance among $Q'$ and the missing answers $\mathcal{D}^-$ as $\argmin_{Q'\in \mathcal{D}} \max\{\Delta(Q',G)|G \in D^{-}\})$, where given $G_1 = (E_1,V_1, \ell_1), G_2 = (E_2,V_2, \ell_2)$, $\Delta(G_1, G_2) = |E_1| + |E_2| - 2 |\mathsf{MCS}(G_1,G_2)|$. 


%% SKYLINE? Do we add this? 
% \eetitle{Skyline queries on graphs.}
% Skyline queries~\cite{zheng2014efficient,zou2010dynamic}  












% \stitle{Exploratory Graph Analysis} 
% 			\begin{itemize}
% 				\item Approximate search:
% 					% \begin{itemize}
% 					% 	\item Structural preserving: homomorphism~\cite{fan2010graph}, strong simulation~\cite{ma2014strong}
% 					% 	\item Incompletely specified patterns~\cite{yang2014slq,khan2013nema,yuan2012efficient}
% 					% \end{itemize}
% 				\item By example paradigm:
% 					\begin{itemize}
% 						\item Graph query by example~\cite{mottin2014exemplar,jayaram2015querying}	
% 						\item Learning paths~\cite{bonifati2014learning} 
% 					\end{itemize}
% 			\end{itemize}
% \stitle{Refinement of Graph Query Results}
% 			\begin{itemize}
% 				\item Reformulation and refinement:
% 					\begin{itemize}
% 						\item Graph Query Reformulation with diversity~\cite{mottin2015graph}
% 						\item Why-empty and Why-so-many results~\cite{vasilyeva2016answering}
% 						\item Result summarization~\cite{ranu2014answering,wu2013summarizing}
% 					\end{itemize}
% 				\item Top-k results: 
% 					\begin{itemize}
% 					 	\item Diversified Top-k Graph Pattern Matching~\cite{fan2013diversified}
% 					 	\item Learning to rank from user-feedback~\cite{su2015exploiting}
% 					 	\item Top-K interesting subgraph discovery in information networks~\cite{gupta2014top,jin2015querying}
% 					 \end{itemize}
% 				\item Skyline queries~\cite{zheng2014efficient,zou2010dynamic}  
% 			\end{itemize}





\section{Key Applications}

\section{Future Directions} 



%This document is intended as a template and guide for the preparation of articles to an encyclopedia, using latex. Contributions should in general follow the usual scheme, "Synonyms, Definitions, Main text (split into various sections with heads and subheads chosen by authors), Conclusions, Cross-references and References", although circumstances might indicate a deviation from this. 
%{\bf Footnotes should not be used}!




% \eat{
% \subsection{Equations}
% \label{subsec:1}
% Use the standard \verb|equation| environment to typeset your equations, e.g.
% %
% \begin{equation}
% a \times b = c\;,
% \end{equation}
% %
% however, for multiline equations we recommend to use the \verb|eqnarray| environment.
% \begin{eqnarray}
% a \times b = c \nonumber\\
% \vec{a} \cdot \vec{b}=\vec{c}
% \label{eq:01}
% \end{eqnarray}

% \subsection{Subsection Heading}
% \begin{quotation}
% Please do not use quotation marks when quoting texts! Simply use the \verb|quotation| environment -- it will automatically render Springer's preferred layout.
% \end{quotation}

% \section{Lists}
% For typesetting numbered lists we recommend to use the \verb|enumerate| environment -- it will automatically render Springer's preferred layout.

% \begin{enumerate}
% \item{Livelihood and survival mobility are oftentimes coutcomes of uneven socioeconomic development.}
% \begin{enumerate}
% \item{Livelihood and survival mobility are oftentimes coutcomes of uneven socioeconomic development.}
% \item{Livelihood and survival mobility are oftentimes coutcomes of uneven socioeconomic development.}
% \end{enumerate}
% \item{Livelihood and survival mobility are oftentimes coutcomes of uneven socioeconomic development.}
% \end{enumerate}

% \paragraph{Paragraph Heading} %
% For unnumbered list we recommend to use the \verb|itemize| environment -- it will automatically render Springer's preferred layout.

% \begin{itemize}
% \item{Livelihood and survival mobility are oftentimes coutcomes of uneven socioeconomic development, cf. Table~\ref{tab:1}.}
% \begin{itemize}
% \item{Livelihood and survival mobility are oftentimes coutcomes of uneven socioeconomic development.}
% \item{Livelihood and survival mobility are oftentimes coutcomes of uneven socioeconomic development.}
% \end{itemize}
% \item{Livelihood and survival mobility are oftentimes coutcomes of uneven socioeconomic development.}
% \end{itemize}

% \section{Tables}
% All tables should have accompanying legends, and corresponding in-text citations need to be provided. A table legend should begin with "Table" (not abbreviated), followed by the number, both in boldface.
% The number is not followed by a period, and the legend has no end-punctuation: Table \ref{tab:1}.
% %
% \begin{table*}
% \caption{Please write your table caption here}
% \label{tab:1}       % Give a unique label
% %
% % Follow this input for your own table layout
% %
% \begin{tabular}{p{2cm}p{2.4cm}p{2cm}p{4.9cm}}
% \hline\noalign{\smallskip}
% Classes & Subclass & Length & Action Mechanism  \\
% \noalign{\smallskip}\svhline\noalign{\smallskip}
% Translation & mRNA$^a$  & 22 (19--25) & Translation repression, mRNA cleavage\\
% Translation & mRNA cleavage & 21 & mRNA cleavage\\
% Translation & mRNA  & 21--22 & mRNA cleavage\\
% Translation & mRNA  & 24--26 & Histone and DNA Modification\\
% \noalign{\smallskip}\hline\noalign{\smallskip}
% \end{tabular}
% $^a$ Table foot note (with superscript)
% \end{table*}
% %
% \section{Figures}
% Color figures can be submitted. The print and electronic publication of the encyclopedia will be in full color. Please submit figures and supplementary materials in their original program format. See also Fig.~\ref{fig:1}.\footnote{If you copy
% text passages, figures, or tables from other works, you must obtain
% \textit{permission} from the copyright holder (usually the original
% publisher). Please enclose the signed permission with the manucript. The
% sources must be acknowledged either in the
% captions, as footnotes or in a separate section of the book.}


% % For figures use
% %
% \begin{figure}
% % Use the relevant command for your figure-insertion program
% % to insert the figure file.
% % For example, with the graphicx style use
% %\includegraphics[scale=.65]{figure}
% %
% % If no graphics program available, insert a blank space i.e. use
% %\picplace{5cm}{2cm} % Give the correct figure height and width in cm
% %
% \caption{Sample figure.}
% \label{fig:1}       % Give a unique label
% \end{figure}

% \section{Definitions}
% If you want to list definitions, we recommend to use the Springer-enhanced \verb|description| environment -- it will automatically render Springer's preferred layout.

% \begin{description}[Type 1]
% \item[Type 1]{That addresses central themes pertainng to migration, health, and disease. Wilson discusses the role of human migration in infectious disease distributions and patterns.}
% \item[Type 2]{That addresses central themes pertainng to migration, health, and disease. Wilson discusses the role of human migration in infectious disease distributions and patterns.}
% \end{description}

% \subsubsection{Theorems}

% \begin{theorem}
% Theorem text goes here.
% \end{theorem}
% %
% % or
% %
% \begin{definition}
% Definition text goes here.
% \end{definition}

% \begin{proof}
% %\smartqed
% Proof text goes here.
% \qed
% \end{proof}

% \paragraph{Paragraph Heading} %
% %
% % For built-in environments use
% %
% \begin{theorem}
% Theorem text goes here.
% \end{theorem}
% %
% \begin{definition}
% Definition text goes here.
% \end{definition}
% %
% \begin{proof}
% \smartqed
% Proof text goes here.
% \qed
% \end{proof}
% \section{Other Options}
% \runinhead{Run-in Heading Boldface Version} Use the \LaTeX\ automatism for all your citations.

% \subruninhead{Run-in Heading Italic Version} Use the \LaTeX\ automatism for all your citations.

% \section{Submission of your article} To submit, login at \url{http://meteor.springer.com} with the user/password you should have received from Springer. Please upload the source files required for compilation (.tex, figures, and .bbl if you use bibtex) as well as a {\bf .pdf file of the compiled document}.

% \section{Cross-References}
% Please login to Meteor (\url{http://meteor.springer.com}) and download a current table of contents. Include a list of related entries from the encyclopedia in this cross-reference section that may be of further interest to your readers. 
% \section{Citations}
% Citations are in NameYear style using natbib citation commands like {\bf \textbackslash citep\{\}} and {\bf \textbackslash citet\{\}}. The basic \textbackslash cite command works identical to \textbackslash citet. \\
% Some examples of citations are given below:

% \begin{itemize}
% \item[-]{Journal article: \citep{Smith99}}
% \item[-]{Book chapter \citep{Aron01} }
% \item[-]{Book, authored: \citep{Brown01}  }
% \item[-]{Proceedings, with an editor:  \citep{Boisnard06}  }
% \item[-]{PhD Thesis: \citep{AlmenaraThesis10} } 
% \end{itemize}
% }

% For bibtex users:
% For references use the `Springer Basic Style'. 
\bibliographystyle{spbasic}  %for bibtex
\bibliography{paper} %for bibtex-example

%For non-Bibtex users:
%\begin{thebibliography}{99}
%\bibitem[Aron 2001]{Aron01} Aron J, Blass B (2001) The future of modern genomics. Blackwell, London
%\bibitem[Brown 2001]{Brown01} Brown B, Aaron M (2001) The politics of nature. In: Smith J (ed) The rise of modern genomics, 3rd edn. Wiley, New York, p 234 -295 
%\bibitem[Smith 1999] {Smith99} Smith J, Jones M Jr, Houghton L et al (1999) Future of health insurance. N Engl J Med 965:325 -329  
%\bibitem[South 1999]{South99} South M (1999) The future of genomics. In: Williams H (ed) Proceedings of the genomic researchers, Boston, 1999
%\end{thebibliography}

\end{document}